\documentclass[11pt]{article}
\usepackage[pdftex]{graphicx} 
\usepackage[comma,numbers,sort&compress]{natbib} % round/square, authoryear/numbers
\usepackage{amsfonts,amsmath,amssymb,amsthm} % various math stuff
% \usepackage[compact]{titlesec}  % Allow compact titles
\usepackage{sectsty} % Allows for section style to be manipulated
\usepackage{wrapfig}  % Allow wrapping of text around figures
\usepackage{mdwlist}  % Make list items closer together: itemize*, enumerate*
\usepackage{sidecap}  % For putting caption beside figure
\usepackage{url} % for web links

% ==  FONT  ===========================
% % ---- uncomment for Palatino -------------
% \usepackage{palatino} 
% ---- uncomment for Helvetica (sansserif)  -------------
\usepackage[T1]{fontenc}
\usepackage[scaled]{helvet}
\renewcommand*\familydefault{\sfdefault}  % If the base font of the document is to be sans serif

% ======= File locations =================
\input{jpdefs} % define some useful latex commands
%\newcommand{\figdir}{figs}  % fig directory

% ---- Control paragraph spacing and indenting  -------------------
\setlength{\textheight}{9in} % 11 - ( 1.5 + \topmargin + <bottom-margin> )
\setlength{\textwidth}{6.5in} % 8.5 - 2 * ( 1 + \oddsidemargin )
\setlength{\topmargin}{ -0.5in}  % in addition to 1.5'' standard margin
\setlength{\oddsidemargin}{0in} % in addition to 1'' standard
\setlength{\parindent}{0em}  % indentation 
\setlength{\parskip}{1ex} % paragraph gap 
\setlength{\headsep}{0.0in}
\setlength{\headheight}{0.0in}
% \raggedbottom  % allows pages to have differing text height 
\allowdisplaybreaks  % allow multi-line equations to continue on the next page

% % ----- Control for sectioning commands -------------------------
% \renewcommand\baselinestretch{0.94}\large\normalsize
% \titlespacing{\section}{0pt}{3pt}{-.5\parskip}   % Margin; Above; Below
% \titlespacing{\subsection}{0pt}{*0}{-.5\parskip}
% \titlespacing{\subsubsection}{0pt}{*0}{-.5\parskip}
% \renewcommand{\refname}{References Cited}
% \sectionfont{\large} 
% \subsectionfont{\normalsize} 
% \subsubsectionfont{\it}

% ----- Float (figure) positioning --------------------------------------
\setcounter{topnumber}{1}  % max # floats at top of page
\setcounter{bottomnumber}{1} % max # at bottom of page
\setcounter{totalnumber}{1} % max floats on a page
\renewcommand{\topfraction}{1} % max frac for floats at top
\renewcommand{\bottomfraction}{1} % max for floats at bottom
\renewcommand{\textfraction}{0} % minimum fraction of page for text
%\setlength{\textfloatsep}{.2in}  % separation between figure and text


% -----  New definitions ------------------
\newcommand{\figref}[1]{Fig.~\ref{fig:#1}}  % use for citing figs
\renewcommand{\eqref}[1]{eq.~\ref{eq:#1}}

% -----  Math symbols --------------------
\newcommand{\LL}{\mathcal{L}}
\newcommand{\vr}{\sigma^2_r}
\newcommand{\vm}{\sigma^2_m}
\newcommand{\vg}{\sigma^2_g}
\newcommand{\va}{\sigma^2_a}
\newcommand{\vv}{\mathbf{v}}
\newcommand{\vx}{\mathbf{x}}
\newcommand{\vecr}{\mathbf{r}}
\newcommand{\vecg}{\mathbf{g}}
\newcommand{\vecm}{\mathbf{m}}
\newcommand{\veca}{\mathbf{a}}
\newcommand{\matA}{\mathbf{A}}
\newcommand{\matD}{\mathbf{D}}
\newcommand{\veps}{\mathbf{\epsilon}}
\newcommand{\matC}{\mathbf{C}}
\newcommand{\vecb}{\mathbf{b}}
\newcommand{\vecx}{\mathbf{x}}
\newcommand{\matM}{\mathbf{M}}
\newcommand{\tA}{\tilde{\mathbf{A}}}
\newcommand{\eye}{\mathbf{I}}
\newcommand{\margmu}{\tilde{\mathbf{\mu}}}
\newcommand{\matSig}{\mathbf{\Sigma}}
% ====================================================
\begin{document}
% % ----- Math spacing commands that have to be within document ----------
% \abovedisplayskip=2pt  % spacing above an equation
% \belowdisplayskip=2pt  % spacing below an equation
% %\abovedisplayshortskip=-5pt
% % \belowdisplayshortskip=-5pt
% % --------------------------------------------------------------


% % --- Publication note ---------
%\vbox to 0.2in{ \vspace{-.3in}
% \small\tt  Draft Information
% \vfil}

% Title
\title{Fragments from: Notes on motion artifact removal in C Elegans data}
% \author{Jonathan W. Pillow}

\maketitle

\setspacing{1.1}
\hbox to \textwidth{\hrulefill} 
\vspace{-.1in}
% Abstract
Some notes on how to model the motion artifacts
in GCaMP fluorescence measurements. (And, hopefully, remove them!) \\
\hbox to \textwidth{\hrulefill}
%--------------------------------------------------------------------------

\vspace{.5in}


\section{OLD STUFF (no longer up to date)}

I believe this stuff all applies if we assume a flat prior over
$m(t)$, i.e., $\vm = \infty$.

We can regard the conditional distribution of the motion artifact as:
\begin{equation}
m(t)\, |\, r(t)  \sim \Nrm(r(t), \vr).
\end{equation}
This allows us to marginalize over $m(t)$ using (\eqref{GCaMP}) and
the appropriate ``Gaussian fun facts", giving:
\begin{equation}
g(t)\, |\, r(t),a(t)  \sim \Nrm\Big(a(t)r(t),\, a(t)^2 \vr + \vg\Big).
\end{equation}
This is everything we need to do maximum likelihood inference for
$a(t)$, the quantity of interest.  I will drop the time index since
the current model treats each time bin independently.  (This is
something we might wish to relax later, since presumably both
activity-related fluorescence and motion artifact have time
dependencies).  

Neglecting $a$-independent constants, we have log-likelihood:
\begin{equation}
\LL \triangleq \log p(g|r,a)  = - \tonehlf \log (a^2 \vr + \vg) - \tonehlf
\frac{(g-ar)^2}{a^2\vr + \vg}.  \label{eq:logli}
\end{equation}


\subsection{Neglecting RFP noise}
Just for interest's sake, let's start by considering what happens when
either noise variance goes to zero.  When the assumed RFP-related
noise variance is zero, $\vr=0$, then we're assuming the RFP
measurements give us perfect access to the motion artifact, and in
this case we recover the divisive correction that the Leifer lab is
currently using.  Neglecting constants, we'd have:
\begin{equation}
\LL = \frac{-(g-ar)^2}{2\vg}
\end{equation}
which implies, obviously:
\begin{equation}
\hat a_{ML} = \frac{g}{r}
\end{equation}

\subsection{Neglecting GCaMP noise}
Ok, that's not so interesting.  Let's look at what happens if we
instead neglect noise in the GCaMP signal, so $\vg = 0$. Then we get
(again, neglecting terms that don't involve $a$):
\begin{equation}
\LL =  -\log (a)   -  \frac{(g-ar)^2}{2a^2\vr}
\end{equation}
Differentiating this and setting to zero gives us:
\begin{equation}
\frac{\partial}{\partial a} \LL  =  -\frac{1}{a}  + \frac{g^2}{a^3
  \vr} - \frac{gr}{a^2\vr}  = 0,
\end{equation}
which (if we multiply by $-a^3 \vr$) implies
\begin{equation}
a^2 \vr + a (gr) - g^2 = 0,
\end{equation}
which has solution
\begin{equation}
\hat a_{ML} =  g \left( \mathrm{sign}(r) \frac{\sqrt{r^2 + 4\vr} - |r|}{2\vr}\right).
\end{equation}
(Note that it takes a little bit of extra work to see that we need the
$\mathrm{sign}(r)$ and absolute value $|r|$, but I'm pretty sure this
is right---it gives the intuitively correct answer, which is that the
inferred $a$ has the same sign as $g/r$). 

And although it isn't obvious, it's reassuring to see that if you take
the limit of this solution as $\vr \rightarrow 0$, you get back the
same solution, $\hat a = g/r$. For finite $\vr$, the estimated $\hat
a_{ML}$ lies between $g/r$ and 0, meaning it's shrunk towards zero by
an amount that depends on the assumed noise variance $\vr$. And,
(happily) the estimate doesn't go to infinity at $r=0$, taking instead
a maximum (absolute) value  $|g|/\sigma_r$.

\subsection{Incorporating both noise terms}

When we include both noise sources, then differenting the
log-likelihood (\eqref{logli}) looks like it's going to give us a
third-order polynomial (but I haven't checked carefully).  Luckily,
these also have an analytic solution (\url{https://en.wikipedia.org/wiki/Cubic_function#General_formula_for_roots})
but perhaps someone wants to use Mathematica to check this (ask Adam
for help if necessary!) rather than wasting the paper?



%-------------------------------------------------------------------------- 
%\bibliographystyle{apalike}
%\bibliography{bibfile.bib}

\end{document} 
